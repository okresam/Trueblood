\chapter{Zaključak i budući rad}
		
		Naš zadatak bio je izraditi banku krvi koju bi primarno koristili donori i djelatnici banke.
	
	Ideja banke krvi i upravljanje količinama krvi je sveprisutna u svijetu, a dostupna je i na mrežnim stranicama \underline{HZTM}\footnote{\url{http://hztm.hr/hr/content/22/zalihe-krvi/831/zalihe-krvi}}.
\\\\
	Projekt smo proveli u tri faze:
	\begin{enumerate}
		\item početna razrada funkcionalnih i nefunkcionalnih zahtjeva
		\item implementacija zahtjeva i daljnje razrađivanje
		\item testiranje i završno dokumentiranje
	\end{enumerate} 

	Na početku prve faze upoznavali smo ideju aplikacije, razrađivali smo funkcionalne zahtjeve i dogovarali se što sve aplikacija može i treba raditi. U prvoj fazi smo se također počeli upoznavati s tehnologijama koje smo koristili. Ni jedan od članova tima nije bio u potpunosti upoznat s tim tehnologijama. Naučili smo da u stvarnosti, prije izrade samog projekta, potrebno je naučiti kako učiti alate, a zatim i naučiti sam alat.
	
	U drugoj fazi projekta krenuli smo sa implementacijom, što je na početku teklo jako sporo jer još nismo bili upoznati s alatima, no kako je vrijeme prolazilo tako smo postajali bolji i implementirali smo zahtjeve sve brže. Također, u ovoj fazi uvidjeli smo propuste u definicijama funkcionalnih i nefunkcionalnih zahtjeva, pa smo naučili biti precizniji u svojim definicijama i uzimati više slučajeva u obzir. 
	
	U trećoj, odnosno posljednjoj fazi, nakon implementacije programskog rješenja, dovršili smo dokumentaciju projekta. Ovaj korak nam je dodatno pomogao u shvaćanju obujma našeg projekta, i ukazao nam na arhitekturalne probleme koje nismo predvidjeli. Također, testiranjem smo ustanovili da sustav ima par grešaka.
	
	Članovi tima su prije projekta bili upoznati s Javom i NodeJS-om, pa bi odabirom tih tehnologija umjesto Reacta pomogao pri ubrzanju razvoja aplikacije. 
	
	Naučili smo važnost koordinacije i komunikacije među članovima tima. Nekad članovi nisu bili dobro koordinirani pa bi jedni implementirali funkcionalnost na ovaj, a drugi na onaj način. Naučili smo meke vještine povezane s projektima općenito, kao što je korištenje alata \texttt{git}, pisanje značajnih commit poruka, korištenje GitLab-a, te poštivanje unaprijed definiranih obrazaca programiranja. 
	
	Najviše od svega, naučili smo značenje vremenske koordiniranosti u grupnom radu. Projekti poput ovoga vremenski su zahtjevni i ponekad se može dogoditi da jedan modul programskog rješenja kasni za drugim, pa drugi ne može nastaviti svoj razvoj. U ovakvim slučajevima, iznimno je važno vremenski se uskladiti unutar tima kako bi se takve neučinkovitosti izbjegle. 
	
	Od definiranih funkcionalnih zahtjeva uspjeli smo implementirati sve koju su bili predstavljeni zadatkom.
	
	U budućnosti bismo mogli ažurirati aplikaciju i nadodati neke nove stvari. Primjerice, mogli bi donirati ljudi iz nekih drugih zemalja, pa bi aplikaciju preveli na engleski jezik. Također dodali bismo da djelatnik i administrator mogu biti ujedno i donori, što trenutno nije slučaj.

	Zaključno, zahvaljujući iskustvima i znanjima koja smo stekli na ovom projektu, kad bismo krenuli iz početka, napravili bismo mnogo bolji projekt mnogo brže.
	\eject 
